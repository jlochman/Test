\documentclass[a4paper,11pt,oneside]{report}
\usepackage[cp1250]{inputenc}  
\usepackage[T1]{fontenc}
\usepackage{graphicx}
\usepackage{url}
%\usepackage{hyperref}
\usepackage{nohyperref}
\usepackage{multirow}
\usepackage{amsthm}
\usepackage{setspace}

\makeatletter
\def\@makechapterhead#1{%
  \vspace*{50\p@}%
  {\parindent \z@ \raggedright \normalfont
    %\ifnum \c@secnumdepth >\m@ne
    %    \huge\bfseries \@chapapp\space \thechapter
    %    \par\nobreak
    %    \vskip 20\p@
    %\fi
    \interlinepenalty\@M
    \Huge \bfseries #1\par\nobreak
    \vskip 40\p@
  }}
  \makeatother

\pdfminorversion=4

\newtheoremstyle{named}{}{}{\slshape}{}{\bfseries}{:}{.5em}{\thmnote{#3}}
\theoremstyle{named}
\newtheorem*{namedtheorem}{Theorem}

\begin{document}

\thispagestyle{empty}

\begin{center}

  \vspace*{40mm}

  {\huge \bf Software Manual for PM-04n}

  \vfill

  {\large
   \begin{tabular}{ll}
     Author:    & Ing. Jan Lochman \\
     Email:     & jan.lochman@icloud.com \\
     Telephone: & +420 604 658 921 \\
     Company:   & VEZAS s.r.o. \\
     Written:   & \today
   \end{tabular}
  }
\end{center}

\clearpage
\pagenumbering{roman}
\setcounter{page}{1}

\begin{spacing}{0.9}
  \tableofcontents
\end{spacing}

\newpage
\clearpage
\pagenumbering{arabic}
\setcounter{page}{1}

\chapter*{Introduction}
\addcontentsline{toc}{chapter}{Introduction}
\markboth{INTRODUCTION}{}

This text describes the software for PM-04n measuring instrument. Software is
designed for controlling two PM-04n's simultaneously. Technically, in software,
there are two almost independent windows for each PM-04n instrument. These
windows share some configuration parameters and database, which stores the saved
measurements, but the measuring procedures are completely separate.

In the first Chapter, the hardware will be presented. This includes the
requirements on the computer running the program and basic description of two
regulators which are inside the PM-04n instrument, together with description of
their communication with PC.

The program is described in the second Chapter. Here, some remarks on
installation are presented. There are some built-in constants, which user is not
allowed to change - these constants will be presented in this Chapter as well.
The biggest part of this Chapter is a description of graphical blocks of the
program, so called GUI (graphical user interface).

In the last Chapter, the instructions for measuring and calibration procedures
are presented. These should provide an user everything, what he needs, to operate
this software.


\chapter{Hardware}
\label{ch:Hardware}

\section{Computer}

Program is compiled as a 32-bit Windows application. It was successfully tested
on both 32 and 64-bit versions of Windows XP, Windows 7 and Windows 8.1. It is
not possible to run this program on different OS, such as Linux or OSX, without
a proper Windows emulator.

PC requirements depend on the version of Windows, you are running on PC,
where you want this program to be installed. Generally, 2 GHz CPU, 2 GB RAM,
1 GB Hard Drive free space, 256 MB Graphics Memory and Ethernet Connection
should be enough for fluent run of the program, delivering enough disc space for
storing thousands of measurements. 

\section{Regulators}

PM-04n instrument contains two regulators, through which the program reads the
instrument status and controls the measurement. Their description follows in
this Section.

\subsection{Eurotherm e3508}
\label{sec:Eurotherm}

The first of these regulators, which can be seen on the front panel of the
PM-04n instrument, is Eurotherm e3508 (Figure~\ref{fig:e3508Front}). The purpose
of this regulator, is to control the furnace temperature, including the turning
on/off the cooling ventilator and safety insurance to ensure the temperature in
the furnace does not exceed the maximum allowed value. 

\begin{figure}[t]
  \centering
  \includegraphics[scale=1]{Figures/Eurotherm.jpg}
  \caption{Eurotherm e3508 front view}
  \label{fig:e3508Front}
\end{figure}

Rare view of this regulator is shown in Figure~\ref{fig:e3508Rare}.
Table~\ref{tab:EurothermConnections} shows the electrical connections, which are
in PM-04n instrument in usage, together with their purpose. Safety relay turns
the heating off, when the temperature in furnace exceeds $600\,^{\circ}{\rm C}$.
The meaning of the remaining connections should be apparent. 

Eurotherm has its own program, independent on the PC. This program
controls, e.g., the turning ventilator on/off, turning the safety relay on/off or PID
regulation of furnace heating which ensures, the desired temperature is reached
effectively.

\begin{table}
  \centering
  \begin{tabular}{c|c}
    \hline
    Connection  & Usage \\
    \hline
    1A:1B        & Furnace \\
    2A:2B        & Ventilator \\
    V+:V-        & Thermocouple \\
    AB:AC        & Safety relay \\
    HA..HF       & Ethernet \\
    \hline
  \end{tabular}
  \caption{Eurotherm electrical connection.}
  \label{tab:EurothermConnections}
\end{table}

\begin{figure}[t]
  \centering
  \includegraphics[scale=0.5]{Figures/EurothermRear.png}
  \caption{Eurotherm e3508 rare view}
  \label{fig:e3508Rare}
\end{figure}

Program reads/writes from/to e3508 the values on addresses shown in
Table~\ref{tab:EurothermAddresses}. Here, ramp units are chosen to be
$^{\circ}{\rm C}\,/\,{\rm min}$ and ramp speed defines a speed, by which
the temperature in furnace should be increased until the desired temperature is
reached. 

\begin{table}
  \centering
  \begin{tabular}{c|c}
    \hline
    Address  & Meaning \\
    \hline
    $2$      & Reading thermocouple temperature \\
    $3$      & Writing desired temperature  \\
    $36$     & Writing ramp speed \\
    $532$    & Writing ramp units \\
    \hline
  \end{tabular}
  \caption{Eurotherm addresses used in the program.}
  \label{tab:EurothermAddresses}
\end{table}

More information about this regulator can be found on the official pages of
Eurotherm Company \url{http://www.eurotherm.com}. On these pages, the basic
software package, caled \textit{iTools}, for Eurotherm devices can be
downloaded. 

\subsection{LabJack T7}
\label{sec:LabJack}

The second regulator, which can be found in PM-04n instrument, is LabJack~T7
(Figure~\ref{fig:LabJackT7}). This regulator is hidden inside the PM-04n
instrument and its purpose is to control the measuring head. This includes the
reading the position of the head, controlling its movement up/down and the most
importantly, LabJack controls the rotational motor and measures the disc
rotational speed. 

\begin{figure}[t]
  \centering
  \includegraphics[scale=0.75]{Figures/LabJack.png}
  \caption{LabJack T7}
  \label{fig:LabJackT7}
\end{figure}

Connections, which are used, together with their usage description, are shown in
Table~\ref{tab:LabJackConnection}. Rotational speed of the head is controlled by
the voltage of the DC motor.

\begin{table}
  \centering
  \begin{tabular}{c|c}
    \hline
    Connection  & Usage  \\
    \hline
    DAC0      & Writing DC motor voltage \\
    CIO2      & Reading and Reseting counter \\
    EIO0      & Reading, if the head is down \\
    EIO1      & Writing to head goes down \\
    EIO2      & Reading, if the head is up \\
    EIO3      & Writing to head goes down \\
    EIO5      & Writing to turn the rotation on/off \\
    \hline
  \end{tabular}
  \caption{Labjack Connections and addresses used in the program.}
  \label{tab:LabJackConnection}
\end{table}

More information about this regulator can be found on the official pages of
LabJack Company \url{http://labjack.com}. On these pages, the software
package can be downloaded under the LabJack~T7 bookmark. To connect to LabJack~T7, the
\textit{Kipling} program serves. 

\section{Communication}
\label{sec:Communication}

Communication is done using TCP/IP protocol. Each device, namely PC,
Eurotherm and LabJack, must have defined its own unique IP address and all
devices have to be in local area network. Program is designed to be able to
control two PM-04n instruments simultaneously, so be sure, each device has its
own unique IP address. 

IP address of Eurotherm, can be found/changed using the official software of
Eurotherm Company, which was mentioned at the end of the Section
\ref{sec:Eurotherm}. LabJack IP address can be found/changed with the official
software of the LabJack Company - Section \ref{sec:LabJack}. Predefined
addresses are shown in Table~\ref{tab:IPAddresses} - please, change them with
caution. 

\begin{table}
  \centering
  \begin{tabular}{|c|c|c|}
    \hline
    Instrument  & Device & IP  \\
    \hline \hline
                              & PC & 192.168.1.2 \\
    \hline
    \multirow{2}{*}{PM-04n 1} & Eurotherm & 192.168.1.233 \\
                              & LabJack   & 192.168.1.208 \\
    \hline
    \multirow{2}{*}{PM-04n 2} & Eurotherm & 192.168.1.234 \\
                              & LabJack   & 192.168.1.209 \\
    \hline
  \end{tabular}
  \caption{IP addresses.}
  \label{tab:IPAddresses}
\end{table}








\chapter{Program}
\label{ch:Program}

\section{Installation}

There is no installation package for PM-04n controlling software. There are only
a few files, which have to be copied into selected directory on hard drive (I
recommend using C:/Program Files/Plastometer/*). Please, make sure, that users
have read/write/modify privileges in that directory to avoid the data loss.

\subsection{File System}

Files needed by program are shown in Figure~\ref{fig:FileSystem}.

\begin{figure}[t]
  \centering
  \includegraphics[scale=0.75]{Figures/FileSystem.png}
  \caption{File System}
  \label{fig:FileSystem}
\end{figure}

\begin{itemize}
  \item \textbf{libmbusmaster.dll, sqlite3.dll} \\
    These are the external libraries which program needs to read/save data and
    to communicate with Eurotherm regulator.
  \item \textbf{marklar.db} \\
    Stores all measurements. To create back-up of measured data, please, store
    this file. 
  \item \textbf{Plastometr.exe} \\
    Executable file by which the program is started. 
  \item \textbf{Plastometr.ini} \\
    Configuration file, which stores the configuration. 
  \item \textbf{langauge/CZ.ini, EN.ini, GE.ini} \\
    Different language versions of program are loaded from these files.
  \item \textbf{language/Translator.exe} \\
    Little application for simple translation between different language
    versions.
\end{itemize}

\section{Built-in Constants, Definitions}
\label{sec:BulitInConstants}

Although there are some values, user can change in GUI of the program, there are
also some fixed, unchangeable, built-it constants, which are hidden from user. 

\begin{itemize}
  \item \textbf{MIN\_TEMP\_MOVE = 240 $^{\circ}$C} \\
    When the measurement is running, then the movement of head is blocked by the
    program, if the temperature is below this value. Reason: Stannum in furnace
    could have not be melted and the user could damage the instrument.  
  \item \textbf{MAX\_TEMP = 600 $^{\circ}$C} \\
    When the measurement is running and the temperature ramp is on, this is the
    temperature, at which the measurement will be stopped by the program.
    Reason: Eurotherm regulator has built-in safety relay, which turns the
    heating off at MAX\_TEMP.
  \item \textbf{QUIT\_TEMP = 20 $^{\circ}$C} \\
    When the program is quit and regulators are still connected, this
    temperature is written on regulators. Reason: Instruments may be still on.
    This should ensure, the instrument will cool down.
  \item \textbf{TOLER\_TEMP = 5 $^{\circ}$C} \\
    If the program has to check, if the desired temperature $t_0$ has been
    reached, then this is the allowed tolerance, i.e., if the temperature is in
    interval $(t_0 - {\rm TOLER\_TEMP}, t_0 + {\rm TOLER\_TEMP})$, then 
    $t_0$ is assumed to be reached.  
  \item \textbf{RAMP\_TEMP = 3 $^{\circ}$C/min} \\
    The rate the furnace is increasing its temperature with time during a
    measurement.
  \item \textbf{STAB\_TIME = 600 s} \\
    During measurement, when the retort goes down to the furnace, this is the
    dead time, during which the temperature in furnace has time to stabilize. If it
    stabilizes, then measurement goes on, if not, the measurement stops. 
  \item \textbf{TOLER\_TEST = 50} \\
    When the sample is prepared, the user has chance to perform a test on
    sample, i.e. try to shortly attach voltage three times greater than the voltage
    attached during the measurement. If there is fewer than 50 pulses counted
    during the test, the test is marked as OK.
\end{itemize}

Each partial measurement has as a result 5 constants

\begin{itemize}
  \item \textbf{T1} \\
    Temperature, at which the first pulse was measured.
  \item \textbf{T2} \\
    Temperature, at which the maximum value of pulses per minute was reached.
  \item \textbf{T3} \\
    Temperature, at which the last pulse was measured.
  \item \textbf{fmax} \\
    Maximum value of pulses per minute measured. This occurs at temperature
    \textit{T2}.
  \item \textbf{ftotal} \\
    Total value of pulses measured during measurement. 
\end{itemize}

\section{Graphical User Interface}

In this Section, the basic blocks of the program GUI (graphical user interface)
are described. In can be seen, that most of the following figures, contain the
blue question mark in it. By clicking on this question mark, a little help is
displayed inside the program, which, in the most cases, should answer an user
question. For more concise information about GUI, you are welcome in this
Section. 

\subsection{Top Panel}
\label{sec:TopPanel}

\textit{Top panel} is shown in Figure~\ref{fig:TopPanel}. It can be seen, the
\textit{Top Panel} is divided into two parts. The top part serves for switching
between PM-04n instruments, the bottom part offers some basic features including
saving measurements/connecting to regulators/starting measurement.

Firstly, let's look at the structure of the top part. User can switch between 3
tabs. The first two tabs are named according to the following logic

\begin{itemize}
  \item \textbf{Instrument Name} \\
    The name of the instrument the tab switches to. This can be changed in the
    \textit{Settings} bookmark (see Section~\ref{sec:Settings}). In
    Figure~\ref{fig:TopPanel}, the instruments are named \textit{Left} and
    \textit{Right}.
  \item \textbf{Measurement Number} \\
    Each measurement has its own unique number. Here is shown the number of
    active measurement. If the number is \textit{\#-1}, then the measurement is
    new and not saved yet. After saving, the number is automatically assigned.
  \item \textbf{Connection Status} \\
    \textit{CONNECTED} or \textit{DISCONNECTED}. 
  \item \textbf{Measuring Status} \\
    \textit{MEASURING} or \textit{PAUSED}. In the first case, the measured
    temperature and dd/min are shown also. 
\end{itemize}

\begin{figure}[t]
  \centering
  \includegraphics[scale=0.75]{Figures/TopPanel.png}
  \caption{Top Panel}
  \label{fig:TopPanel}
\end{figure}

The last tab contains only \textit{Instrument 1 Name $|$ Instrument 2 Name} and
should be used during measurement, where I recommend the program
lookup as shown in Figure~\ref{fig:RecommendeSetting}.

The bottom part of the \textit{Top Panel} contains some image buttons, which
have some built-in hint help. These buttons are

\begin{itemize}
  \item \textbf{Create a New Measurement} \\
    A new measurement is prepared. Data are cleared, graph emptied and the
    \textit{Measurement Number} is changed to $-1$. This number changes after
    saving the measurement. 
  \item \textbf{Save/Update Measurement} \\
    The actual measurement is either saved or updated.
  \item \textbf{Create a Protocol and Export Data} \\
    There is a fixed path, where all the protocols and data are exported. This
    path is defined in configuration file \textit{Plastometr.ini} under
    \textit{ProtocolPATH=} and can be changed here. By default, this path is set
    to \textit{/Desktop/Protocols/}. Inside this directory, program creates a
    new directory with the structure

    \textit{MeasurementNumber\_SampleName\_SomeOtherNumber/} 
    
    and adds two files in it. First of these is a protocol in pdf format,
    which is shown in Section~\ref{sec:Protocol} and the second is exported data
    in txt format, which is described in more details in
    Section~\ref{sec:DataExport}.  
  \item \textbf{Create a Connection with Regulators} \\
    Connects to both regulators. There are some messages which should help
    detect a problem, if there is any. If the connection is not successful, try
    to connect again.
  \item \textbf{Disconnect from Regulators} \\
    Connection to both regulators is closed. If program is measuring, then the
    measurement is stopped.
  \item \textbf{Start Measurement} \\
    Starts the measurement and the periodic reading data from instrument. The
    starting temperature (see Section~\ref{sec:MeasurementControlPanel}) is
    automatically written and \textit{Settings} and a\textit{List of
    Measurements} bookmarks are disabled. Note that buttons controlling the head
    movement up/down are enabled only when temperature is above a certain
    threshold (see Section~\ref{sec:BulitInConstants}).  
  \item \textbf{Stop Measurement} \\
    Stops the measurement and the periodic reading data from instrument. The
    head rotation is disabled, \textit{Settings} and \textit{List of
    Measurements} bookmarks are enabled and the buttons controlling the head
    movement up/down are enabled.  
  \item \textbf{Language Buttons} \\
    Switches between different language versions of the program. 
\end{itemize}

\begin{figure}[t]
  \centering
  \includegraphics[scale=0.35]{Figures/RecommendedSetting.png}
  \caption{Recommended program lookup for measuring on two PM-04n's
  simultaneously.}
  \label{fig:RecommendeSetting}
\end{figure}

\subsection{List of Measurements}
\label{sec:ListOfMeasurements}

First of the bookmarks on the left is the \textit{List of Measurements}
bookmark. It is shown in Figure~\ref{fig:ListOfMeasurements} and its purpose is
to search in the saved measurements. Measurement is loaded after mouse click on
it in the table. There are two ways, how to find the desired measurement.  

\begin{figure}[t]
  \centering
  \includegraphics[scale=0.75]{Figures/ListOfMeasurements.png}
  \caption{List of Measurements bookmark}
  \label{fig:ListOfMeasurements}
\end{figure}

\begin{enumerate}
  \item \textbf{According to the Measurement Number} \\
    In this case, you simply enter the measurement number into the
    \textit{Measurement Number} edit. If the measurement with this number
    exists, it is shown in table, if not, the table is left empty.
  \item \textbf{According to the Remaining Parameters} \\
    There are 4 additional parameters, you can search by, namely
    \begin{itemize}
      \item Sample 
      \item Operator
      \item Date (in format \textit{YYYY-MM-DD})
      \item Finished
    \end{itemize}
    In the table, the measurements fulfilling all the criteria, you enter, are
    displayed. In the example in Figure~\ref{fig:ListOfMeasurements}, the
    measurements performed in \textit{2015-07} with operator starting at
    \textit{Loch} are displayed. 
\end{enumerate}

Next, the number of found measurements is shown. You can go to the last
measurement by pressing the \textit{Last Measurement} button. It is also
possible to remove selected measurement by pressing the \textit{Remove
Measurement} button at the bottom.

\subsection{Settings}
\label{sec:Settings}

Next to the \textit{List of Measurements} bookmark, there is the
\textit{Settings} bookmark. This is shown in Figure~\ref{fig:Settings}. This
bookmark is divided into several blocks, each of them has its own built-in help. 

\begin{figure}[t]
  \centering
  \includegraphics[scale=0.75]{Figures/Settings.png}
  \caption{Settings bookmark}
  \label{fig:Settings}
\end{figure}

\begin{itemize}
  \item \textbf{Save Settings} \\
    This button simply saves the current settings. These include not only the
    values displayed in the \textit{Settings} bookmark, but some values shown
    elsewhere in the GUI of the program. These include the \textit{Start} and
    \textit{Finish Temperature} (Figure~\ref{fig:MeasurementControlPanel}) and
    the current axis settings (Figure~\ref{fig:Graph}).
  \item \textbf{Instrument, LabJack IP, Eurotherm IP} \\
    Instrument characterizes the name of the instrument and is displayed at the
    top of the \textit{Top Panel} (Figure~\ref{fig:TopPanel}). Instrument is defined by
    IP addresses of two regulators, which must be entered here.
  \item \textbf{Company Name, Name Supplement} \\
    Are shown in the head of the protocol, as it is demonstrated in 
    Figure~\ref{fig:ProtocolFirstPage}.
  \item \textbf{Data Reading Interval} \\
    Interval, the data are read from the regulators. The pulses per minute are
    displayed and calculated in this interval as well. 
  \item \textbf{Calibration} \\
    Calibration starts by the pressing the \textit{Calibrate} button. After
    this, the voltage shown in the edit is written on the LabJack. Calibration
    is done using the \textit{QWEASD} keys and is stopped either by pressing the
    \textit{Save Settings} button, or by pressing the \textit{ESC} key.
    Calibration is described in more detail in Section~\ref{sec:Calibration}
  \item \textbf{Calibration Table} \\
    Is delivered together with the temperature sensor. The row with values
    $(500,2)$ means, that if regulator shows a temperature $500\,^{\circ}{\rm
    C}$, than the real temperature is about $2\,^{\circ}{\rm C}$ higher. There
    are two buttons over the table allowing adding/removing the selected row in
    table.
\end{itemize}

\subsection{Measurement Information}
\label{sec:MeasurementInformation}

\textit{Measurement Information} bookmark contains the basic informations about the
measurement and is shown in Figure~\ref{fig:MeasurementInformation}. It can be
filled during the measurement. Some of these informations are used to search the
measurement in the \textit{List of Measurement} bookmark described in
Section~\ref{sec:ListOfMeasurements}, remaining informations are used in the
head of the protocol, which is shown in Figure~\ref{fig:ProtocolFirstPage}. 

\begin{itemize}
  \item \textbf{Measurement Record} \\
    Unique measurement number assigned to the measurement. This is done
    automatically by the program after the first measurement save and cannot be
    changed.
  \item \textbf{Operator} \\
    Name of the person performing the measurement. 
  \item \textbf{Sample Name} \\
    Short name of the measured sample.
  \item \textbf{Finished} \\
    If there is no interest in adding additional partial measurement into
    measurement, then \textit{Finished} can be checked.
  \item \textbf{Sample Characteristic} \\
    Additional characteristic of the measured sample.
  \item \textbf{Note} \\
    Additional information e.g. about the non-standard measurement procedure.
\end{itemize}

It is recommended to save the measurement after editing this information. 

\begin{figure}[t]
  \centering
  \includegraphics[scale=0.75]{Figures/MeasurementInformation.png}
  \caption{Measurement Information bookmark}
  \label{fig:MeasurementInformation}
\end{figure}

\subsection{Measured Data}
\label{sec:MeasuredData}

\textit{Measured Data} bookmark contains the data measured during the measurement and is
shown in Figure~\ref{fig:MeasuredData}. Table at the top contains the data
measured during one concrete partial measurement. The partial
measurement, which is displayed, is shown at the top. At the bottom, the results
of partial measurements are shown and from the checked measurements, the
final results are calculated. These are shown at the bottom. The meaning of the symbols
used here, is defined in Section~\ref{sec:BulitInConstants}. More details to each building
block follows.

\begin{figure}[t]
  \centering
  \includegraphics[scale=0.75]{Figures/MeasuredData.png}
  \caption{Measured Data bookmark}
  \label{fig:MeasuredData}
\end{figure}

\begin{itemize}
  \item \textbf{Listing Between Partial Measurements} \\
    Using buttons $<<$ and $>>$. Between these buttons, the
    partial measurement number is shown together with the instrument, which
    this partial measurement performed. 
  \item \textbf{Table Time,Temperature,ddpm} \\
    Shows the measured data for the selected partial measurement. ddpm are the pulses
    per minute. 100 pulses equals one revolt. 
  \item \textbf{Determine T1, Determine T3} \\
    Although T1 and T3 are determined automatically by the program, user can
    change these values by moving the corresponding points on the graph.
    When these buttons are pressed, program determines T1 or T3 again. 
  \item \textbf{Table T1,T2,T3,fmax,ftotal} \\
    Results for partial measurements.
  \item \textbf{Results at the Bottom} \\
    From the checked measurements, the average and variance of measurement
    characteristic values are calculated.
\end{itemize}

In protocol, only checked partial measurements are used. This can be seen
from Figure~\ref{fig:ProtocolFirstPage}.

\subsection{Measurement Control Panel}
\label{sec:MeasurementControlPanel}

This panel is the basic operational block of the program and is shown in
Figure~\ref{fig:MeasurementControlPanel}. It shows the actual values read
from the instrument, allows user to write a temperature, set the measurement
final temperature, manipulate with the measurement head, finish the measurement
and guide the user around the measurement. The description of building
blocks follows.

\begin{itemize}
  \item \textbf{Temperature} \\
    Actual temperature value from regulator. \textit{Start temperature} is the
    temperature at which the measurements starts, \textit{Final temperature} is
    the temperature at which the measurement will be stopped by the program
    automatically.
  \item \textbf{ddpmin/ddTotal} \\
    Actual value of pulses per minute (ddpmin) and total number of pulses
    (ddTotal).
  \item \textbf{Position} \\
    Shows the head position.
  \item \textbf{Move} \\
    Controls the movement of the measuring head. If the measurement is in
    progress, then these buttons are disabled for temperature below a certain
    threshold defined in Section~\ref{sec:BulitInConstants}.
  \item \textbf{Start a New Partial Measurement} \\
    Prepares a new partial measurement inside the opened measurement. 
  \item \textbf{Finish} \\
    Completes the measurement, i.e. forces the measuring head to move up, turns
    the rotation off and writes the starting temperature, so the instrument is
    prepared for a new measurement as soon as possible. 
  \item \textbf{Preparation of Measurement} \\
    Sample preparation. Checks three events:
    \begin{enumerate}
      \item \textbf{Furnace Heated up} \\
        The furnace has to be heated up to the starting temperature. If the
        error is smaller than the allowed tolerance defined in
        Section~\ref{sec:BulitInConstants}, the \textit{OK} is displayed.
      \item \textbf{Sample Test} (ARBITRARY) \\
        User can try to attach shortly a voltage three times greater than the
        voltage applied during the measurement, to test, if the sample is prepared
        correctly. The test starts by pressing the \textit{Test} button and ends
        automatically. Test is assumed to be successful, if a number of pulses
        counted during test is smaller than built-in constant defined in
        Section~\ref{sec:BulitInConstants}. \textit{OK} is then displayed.
      \item \textbf{Sample Position} \\
        When the furnace is heat up, user must put the sample inside the
        furnace, i.e. by pressing the \textit{Move Down} button. If the position
        \textit{Down} is read, then \textit{OK} is displayed.
    \end{enumerate}
  \item \textbf{Stabilization Retort Temperature} \\
    Program waits time defined in Section~\ref{sec:BulitInConstants} to the
    temperature in furnace stabilizes. If it happens, the measurement continues
    automatically after time counter. 
\end{itemize}

\begin{figure}[t]
  \centering
  \includegraphics[scale=0.75]{Figures/MeasurementControllPanel.png}
  \caption{Measurement control panel}
  \label{fig:MeasurementControlPanel}
\end{figure}

\subsection{Graph}
\label{sec:Graph}

The purpose of this block is to graphically display a measured data of one
selected partial measurement. This block is shown in Figure~\ref{fig:Graph} and
you can see, next to the red line representing the measured data, three blue
points are displayed - these are the three characteristic temperature points T1,
T2 and T3 defined in Section~\ref{sec:BulitInConstants}. Above the graph, there
are some options to configure the graph axes. 

\begin{itemize}
  \item \textbf{X Axis} \\
    You can switch between \textit{Absolute} and \textit{Relative} options. In
    \textit{Absolute} case, the axis range is entered manually and graph is
    repainted after pressing \textit{Enter} key. In \textit{Relative} case, the
    axis range is chosen automatically to comprise all the measured data. When
    a protocol is created (see Section~\ref{sec:Protocol}), then all graphs
    in it are set according to this settings.
  \item \textbf{Y Axis} \\
    What holds for the X Axis is holds for Y Axis as well. There is, however,
    one additional option for the Y Axis: You can switch between logarithmic and
    linear scale.
  \item \textbf{Graph} \\
    Red line represents the measured data and three blue-square points represent
    the three significant temperatures T1, T2 and T3 defined in
    Section~\ref{sec:BulitInConstants}. The T1 and T3 can be dragged by the
    mouse, but the T2 is fixed, unchangeable.
\end{itemize}

\begin{figure}[t]
  \centering
  \includegraphics[scale=0.75]{Figures/Graph.png}
  \caption{Graph}
  \label{fig:Graph}
\end{figure}

\subsection{Protocol}
\label{sec:Protocol}

When the \textit{Create a Protocol and Export Data} button (see
Section~\ref{sec:TopPanel}) is pressed, two files are created. First of these
files is the measurement protocol in pdf format, whose structure is described in
this Section, the second file is the exported data in txt format. Its structure
is described in the next Section. The path, where these files are saved, is
defined in Section~\ref{sec:TopPanel}. 

After files are created, the protocol is opened automatically by the default
program for .pdf extension (I recommend the use of \textit{Adobe Acrobat
Reader}, but other programs should serve as well). Using this program, you can
easily print the created protocol. 

The first page of the protocol has the structure shown in
Figure~\ref{fig:ProtocolFirstPage}. 

\begin{figure}[t]
  \centering
  \includegraphics[scale=0.5]{Figures/ProtocolFirstPage.png}
  \caption{Example of the first protocol page.}
  \label{fig:ProtocolFirstPage}
\end{figure}

\begin{figure}[p]
  \centering
  \includegraphics[scale=0.75]{Figures/ProtocolNextPage.png}
  \caption{The second protocol page. The first page of this protocol is shown
    in Figure \ref{fig:ProtocolFirstPage}.}
  \label{fig:ProtocolNextPage}
\end{figure}

\begin{itemize}
  \item \textbf{Title} \\
    Is loaded from the \textit{Settings} bookmark (Section~\ref{sec:Settings}).
  \item \textbf{Information above Results} \\
    Is loaded from the \textit{Measurement Information} bookmark
    (Section~\ref{sec:MeasurementInformation}). Only \textit{Date} is not shown
    in this bookmark, it is loaded directly from the database.
  \item \textbf{Results} \\
    Are loaded from the \textit{Measured Data} bookmark
    (Section~\ref{sec:MeasuredData}). You can see, that only checked partial
    measurements are included in the protocol and that the mean and variance is
    calculated from these measurements only. 
  \item \textbf{Sample, Note} \\
    Are loaded from the \textit{Measurement Information} bookmark
    (Section~\ref{sec:MeasurementInformation}).
\end{itemize}

The second (and any other) page of the protocol has the structure shown in
Figure~\ref{fig:ProtocolNextPage} and generally, it displays the results of
partial measurements shown in the first protocol page.

\begin{itemize}
  \item \textbf{Header} \\
    Partial measurement number together with the results of characteristic
    values. These values are the values shown in one row in the table from
    Figure~\ref{fig:ProtocolFirstPage}.
  \item \textbf{Graph} \\
    Uses the axis settings defined in Section~\ref{sec:Graph}.
\end{itemize}

\subsection{Data Export}
\label{sec:DataExport}

Next to the protocol, the data are exported into separate file. This serves for
further processing in table-oriented programs such as \textit{Microsoft Excel}.
The structure of data exported is shown in Figure~\ref{fig:ExportDat}.

By closer look at the data structure on the left, you can see, the data are
semicolon separated values, which can be easily loaded as table, so you get 
the table structure on the right.

Each partial measurement is displayed on three columns with the structure

\begin{itemize}
  \item \textbf{Finished} \\
    In the first row, there is shown, if the measurement is checked as
    \textit{Finished} or not. This means, also the results of the unchecked
    measurements are exported.
  \item \textbf{Characteristic Values} \\
    Next 5 rows contain the measurement characteristic values defined in
    Section~\ref{sec:BulitInConstants}.
  \item \textbf{Measured Data} \\
    Table containing all the data measured during the partial measurement. 
\end{itemize}

\begin{figure}[t]
  \centering
  \includegraphics[scale=0.75]{Figures/ExportDat.png}
  \caption{Data exported by the program.}
  \label{fig:ExportDat}
\end{figure}







\chapter{Usage}
\label{ch:Usage}

This Chapter describes the basic instructions to perform a measurement and
calibration on PM-04n instrument. Some informations from the previous chapters
are, for convenience, written here as well, but for more information, a
references to previous Sections are provided.  

\section{Measurement}
\label{sec:Measurement}

\subsection{Select the Instrument}

Firstly, it is needed to select the instrument, you are going to measure on.
For this purpose, the top of the \textit{Top Panel} serves. This is shown in
Section~\ref{sec:TopPanel}.

\subsection{Connect to the Regulators}

Next, you must connect program to regulators. This is done using the
\textit{Create a Connection with Regulators} button, which is the fourth button
on the \textit{Top Panel} (Section~\ref{sec:TopPanel}).

If there is any problem with connection, you are welcome to try pressing the
\textit{Create a Connection with Regulators} button a few times (if the PM-04n
instrument was turned on recently, the regulators can not be initialized yet).
Program has built-in messages to help you indicate the problem. When the
connection is successful, the program let you know about the success.

If you are unable connect to regulators, see Section~\ref{sec:Communication} for
more details - probably, there is a problem with the network settings and you will
need an assistance from the network administrator. 

\subsection{Select the Measurement}
\label{sec:SelectMeasurement}

Before the measurement starts, it is needed to choose, where the measurement
will be added. Basically, there are two possibilities

\begin{enumerate}
  \item \textbf{Add a Partial Measurement to an Existing Measurement} \\
    You have a sample, which was already measured and the measurement was
    successfully saved and you want to add to this measurement a new partial
    measurement.

    You must load your measurement using the \textit{List of Measurements} bookmark
    (Section~\ref{sec:ListOfMeasurements}) and press the \textit{Start a New
    Partial Measurement} button in the \textit{Measurement Control Panel}
    (Section~\ref{sec:MeasurementControlPanel}).

  \item \textbf{Start a New Measurement} \\
    You have a new sample, which was not measured yet. By pressing the
    \textit{Create a New Measurement} button in the \textit{Top Panel}
    (it is the first button, Section~\ref{sec:TopPanel}), a new measurement is
    prepared.

    I recommend to fill \textit{Operator} and \textit{Sample Name} in the
    \textit{Measurement Information} bookmark
    (Section~\ref{sec:MeasurementInformation}) and save the measurement using
    the \textit{Save/Update Measurement} button at the \textit{Top Panel}
    (it is the second button, Section~\ref{sec:TopPanel}). 
\end{enumerate}

Generally, you can edit the information in \textit{Measurement Information}
bookmark (Section~\ref{sec:MeasurementInformation}) anytime, even during the
measurement. If you make some changes to it, do not forget to save them.

\subsection{Start the Measurement}

Now, everything is prepared to measurement could start. This is done by pressing
the \textit{Start Measurement} button at the \textit{Top Panel} (it is the
sixth button, Section~\ref{sec:TopPanel}).

By pressing this button, the Start Temperature is automatically written, but
it can be changed easily using the \textit{Send} button in the
\textit{Measurement Control Panel} (Section~\ref{sec:MeasurementControlPanel}).
Please, check, if the \textit{Final Temperature} is set correctly as well.

\begin{namedtheorem}[Note]
  The \textit{List of Measurements} and \textit{Settings} bookmarks are
  disabled, when the measurement is on. To enable them, you must press
  \textit{Stop Measurement} button, which is next to the \textit{Start
  Measurement} button. Use this with a caution.
\end{namedtheorem}

\begin{namedtheorem}[Note]
  The movement of the measuring head is disabled, if the temperature in the
  furnace is below the cutoff defined in Section~\ref{sec:BulitInConstants}.
\end{namedtheorem}

\subsection{Prepare Sample}
\label{sec:PrepareSample}

In this step, no interaction between user and program is needed. After retort is
prepared and placed in the measuring head, continue to next step.

\subsection{Test Sample (Arbitrary)}

The sample test is arbitrary and is performed by pressing the \textit{Test}
button in the \textit{Measurement Control Panel}
(Section~\ref{sec:MeasurementControlPanel}). 

\subsection{Wait for the Furnace is Heated Up}

Wait until the furnace is heated up to the desired starting temperature. This is
checked automatically by the program and user can see this in the
\textit{Preparation of Measurement} part of the \textit{Measurement Control
Panel} (Section~\ref{sec:MeasurementControlPanel}). If there is
\textit{OK} next to the \textit{Furnace heated up}, you can proceed to next
step.

\begin{namedtheorem}[Note]
  Every time, the program checks, if temperature has reached a desired value,
  there is a built-in tolerance defined in Section~\ref{sec:BulitInConstants}.
\end{namedtheorem}

\subsection{Move Sample into the Furnace}

By pressing the \textit{Move Down} button, the measuring head goes down to the
furnace and measurement starts automatically. You are welcome to enter the next
step. 

\begin{namedtheorem}[Note]
  The movement of the measuring head is disabled, if the temperature in the
  furnace is below the cutoff defined in Section~\ref{sec:BulitInConstants}.
\end{namedtheorem}

\subsection{Make a Coffee}

The measurement is now fully controlled by the program. Firstly, there is a fixed time
interval, defined in Section~\ref{sec:BulitInConstants}, which program waits for
temperature to stabilize after the sample entered the furnace. If the temperature
stabilizes, the measurement goes on, if not, the measurement goes to the
Stage~\ref{sec:FinishTheMeasurement}.

After successful stabilization of temperature, program enables the temperature
ramp (fixed value defined in Section~\ref{sec:BulitInConstants}) and sends the
\textit{Final Temperature}. Next to this, the head starts to rotate with the
voltage defined in \textit{Settings} bookmark (Section~\ref{sec:Settings}).

If the \textit{Final Temperature} is reached, the measurement goes to the last
stage. User can evoke this event by pressing the \textit{! Finish !} button in
the \textit{Measurement Control Panel}
(Section~\ref{sec:MeasurementControlPanel}).

\begin{namedtheorem}[Note]
  Every time, the program checks, if temperature has reached a desired value,
  there is a built-in tolerance defined in Section~\ref{sec:BulitInConstants}.
\end{namedtheorem}

\subsection{Finish the Measurement}
\label{sec:FinishTheMeasurement}

Finalization of the measurement. Program moves the measuring head up, disables
the temperature ramp, sends the \textit{Start Temperature} and disables the head
rotation, so the instrument is prepared for next measurement as soon as
possible. Measurement is saved automatically as well.

If you want to make a new partial measurement, proceed to
Stage~\ref{sec:PrepareSample}.

Otherwise, stop the measurement by pressing \textit{Stop Measurement} button
(the last button on the \textit{Top Panel}, Section~\ref{sec:TopPanel}) and
continue to Stage~\ref{sec:SelectMeasurement}.


\section{Calibration}
\label{sec:Calibration}

To perform a calibration, go to the \textit{Settings} bookmark
(Section~\ref{sec:Settings}). Here, in part \textit{Calibration}, the
\textit{Operating Voltage} item is located. This voltage has to be set during the
process of calibration. The way, this voltage is set, is defined by the norm.

To start the calibration, press the \textit{Calibrate} button (make sure, the
program is connected to regulators). This causes the motor to start rotate. Now,
by pressing the \textit{QWEASD} keys, find the voltage which satisfies the
conditions stated by the norm. 

Calibration ends either by pressing \textit{ESC} key (do not forget to save a new
settings thereafter) or by pressing \textit{Save Settings} button. Both of these
actions cause the rotation to stop.

\end{document}
